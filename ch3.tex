\chapter{养老保障体系及商业养老保险}

养老保障制度与商业养老保险的发展,既是人口结构变化的结果,也是经济社会制度演进的体现。人口老龄化程度加深、少子化趋势延续以及预期寿命持续提升,使得退休期长度显著增加,劳动年龄人口与老年人口之间的相对结构发生变化。在这一背景下,养老保障体系需要在保障适度、权责清晰与长期可持续之间形成可检验的制度安排;与此同时,居民养老需求在资金来源、风险偏好与支付方式等方面呈现多样化特征,单一来源的养老供给难以覆盖不同人群的差异化约束条件。

本章以人口老龄化的宏观事实为起点,先行阐释老龄化形成机制与常用指标体系,并在此基础上分析养老需求的结构特征与资金准备逻辑。继而,从国家积极应对人口老龄化的总体部署与金融监管政策导向出发,说明多层次养老保险制度建设的制度含义与行业定位。随后,本章围绕三支柱养老保障体系,分别讨论第一支柱基本养老保险、第二支柱企业年金与职业年金、第三支柱个人养老金的制度框架与运行要点,并对其覆盖范围、资金筹集方式、账户管理机制与可持续性挑战加以归纳。进一步地,本章在第三支柱制度落地的制度背景下,讨论商业保险年金与专属商业养老保险的产品功能、账户结构与给付机制,并结合年金产品定价与盈利性管理框架,说明预定假设、利润测试与三差平衡等精算经营要点。最后,通过养老方式变迁的示例与养老资金数量化演算,展示养老准备在时间维度上的跨期规划特征,为后续精算建模与产品设计章节提供直观的业务语境。

\newpage

\section{人口老龄化与养老需求}

本节将从渐进式延迟法定退休年龄的制度背景出发,结合人口金字塔、出生人口、生育意愿与寿命改善等因素,阐释老龄化加速的形成机制与常用指标体系。老龄化社会向深度老龄化阶段的跃迁、复杂老龄化与抚养比上升共同构成养老保障体系的结构性压力来源。养老资金来源的调查结果显示,居民对养老准备的自我储备倾向显著增强,养老需求呈现多元化与长期化特征。这些事实共同奠定了多层次养老保险制度建设与商业养老保险发展的现实基础。

\subsection{渐进式延迟法定退休年龄与人口结构变化}

\subsubsection{渐进式延迟法定退休年龄的制度含义}

渐进式延迟法定退休年龄是应对人口结构变化的重要制度安排,其核心目的在于在制度层面调整工作期与退休期的时间配置,从而在一定程度上缓解养老金支出压力并增强缴费期的资金积累能力。相关制度安排之所以受到高度关注,原因在于退休年龄的变动会直接改变个人的劳动供给期限与退休待遇领取期限,进而通过收入端与支出端同时作用于基本养老保险基金的收支平衡格局。与国际经验相比,部分发达国家的法定退休年龄通常在65岁及以上,甚至存在更高的退休年龄安排;我国在既有制度结构与人口结构条件下推进渐进式调整,体现了在制度可承受性与社会预期稳定性之间寻求平衡的政策取向。

从基金收支机制看,延迟退休通过减少领取养老金的年数、增加缴费年数与延迟待遇起领时点等路径影响基金现金流。若以退休后领取年限为观察窗口,退休年龄每延后一年,理论上可减少相应年限的待遇支付;同时,劳动者延长工作期意味着缴费基础延长,单位与个人的缴费持续时间随之增加。由此,渐进式延迟法定退休年龄不仅是劳动市场与社会保障制度的协调问题,也与养老保障体系的长期可持续性密切相关。

\subsubsection{人口金字塔的识别方法与分析用途}

人口金字塔以图形方式呈现人口分布情形。其基本绘制规则为以年龄为纵轴、以人口比例为横轴,并按左侧表示男性、右侧表示女性;底部对应低年龄组人口,上部对应高年龄组人口。人口金字塔的主要用途在于预示人口趋势,尤其适用于识别少子化、老龄化与长寿化等现象的形态特征及变化方向。通常而言,不同形态的人口金字塔对应不同的人口增长与年龄结构特征,因而能够为养老保障制度评估提供直观的结构信息。

\begin{figure}[htbp]
  \centering
  \includegraphics[width=0.9\textwidth]{figure/PPT_3_6.png}
  \caption{人口金字塔的基本结构与类型示意。图像来源:演示文稿第6页。}
  \label{fig:pyramid_basic}
\end{figure}

人口金字塔的类型识别强调对底部与顶部形态的综合判断。底部宽而顶部窄的形态通常意味着出生人口规模较大、年轻人口占比较高;底部逐渐收窄而中部或顶部相对扩张的形态则提示出生人口下降、老年人口占比上升。在养老保障分析中,人口金字塔的价值主要体现在两方面:其一,可以直观呈现未来一定时期内劳动年龄人口与老年人口的相对规模变化,从而提示缴费基础与待遇领取规模的可能变动;其二,可以作为预测与情景分析的输入信息,为抚养比变化、基金收支压力评估以及商业养老产品需求扩张提供依据。

\subsubsection{我国人口金字塔的阶段性演变及启示}

我国人口结构随时间推移呈现出明显的阶段性特征。人口金字塔由相对典型的“底宽顶窄”逐步演变为“中部扩张、底部收窄”的形态,并进一步出现顶部逐渐加厚的趋势。这一变化意味着新增劳动力供给的相对增速下降,同时老年人口规模与占比持续上升,养老保障体系的负担结构随之发生改变。

\begin{figure}[htbp]
  \centering
  \includegraphics[width=0.86\textwidth]{figure/PPT_3_7.png}
  \caption{人口金字塔与老龄化趋势示意。图像来源:演示文稿第7页。}
  \label{fig:pyramid_trend}
\end{figure}

人口金字塔的阶段性演变并非单一因素所致,而是出生人口变化、生育意愿变化与寿命改善等多因素共同作用的结果。对养老保障体系而言,劳动年龄人口相对减少将抬升老年抚养负担,并可能导致基本养老保险在现收现付框架下面临更为显著的收支压力;与此同时,退休期延长意味着个体在退休后需要更长时间的养老资金供给,从而推动养老需求从“短期补足”转向“长期规划”,并对养老金融产品的稳健积累、领取灵活性与长期保障功能提出更高要求。

\subsection{老龄化形成机制与指标体系}

\subsubsection{出生人口变化及其结构含义}

出生人口的变化直接影响未来劳动力供给规模与年龄结构的底部形态。根据人口统计资料,我国全国总人口从2000年的约12.67亿增长至2024年的约14.08亿,但出生人口规模在同一时期呈现显著下降:2000年出生人口约1765万,而2024年约954万。出生人口规模下降意味着未来进入劳动年龄段的人群规模趋于收缩,缴费人群增长的基础趋弱;同时,由于人口结构的惯性效应,出生人口下降的影响往往在数十年后集中体现为劳动年龄人口占比下降与老年人口占比上升。

出生人口下降对养老保障的影响主要体现在两个层面。宏观层面上,缴费基础增长放缓将限制基金收入增速;若同时伴随老年人口规模上升,则支出增长可能快于收入增长,从而提高制度可持续性压力。微观层面上,未来家庭子女数量减少会改变家庭内部代际支持的结构条件,使得依赖家庭单一渠道提供养老支持的可行性下降,个体对制度保障与市场化养老金融工具的依赖程度相应提高。

\begin{table}[htbp]
  \centering
  \caption{我国人口结构变化的若干关键指标。数据来源:演示文稿第7--9页。}
  \label{tab:demo_key}
  \renewcommand{\arraystretch}{1.25}
  \begin{tabular}{p{0.34\textwidth}p{0.58\textwidth}}
    \toprule
    指标 & 典型年份与数值 \\
    \midrule
    全国总人口 & 2000年约12.67亿;2024年约14.08亿 \\
    出生人口 & 2000年约1765万;2024年约954万 \\
    妇女总和生育率 & 1980年2.24;2023年1.02 \\
    人均预期寿命 & 2024年约79岁(部分地区约84岁) \\
    \bottomrule
  \end{tabular}
\end{table}

\subsubsection{生育意愿与总和生育率指标}

总和生育率是衡量生育水平的重要指标,反映妇女一生中平均生育的小孩数量。相关统计显示,1980年妇女总和生育率为2.24,2023年为1.02。总和生育率的下降与生育意愿变化密切相关。生育意愿受到经济社会因素、家庭成本结构与生活方式等多重影响,其变化将通过较长的时间链条影响人口金字塔底部,从而加速人口老龄化进程。少子化被视为老龄化的“加速器”,其原因在于出生人口的持续下降会在相对较短时间内抬升老年人口占比,即使老年人口绝对数量并未出现同等幅度增长,人口结构的相对变化仍会显著改变抚养压力。

\subsubsection{寿命改善与长寿时代特征}

随着生活水平提高与医疗卫生条件改善,我国人均预期寿命显著提高。相关资料显示,2024年我国人均预期寿命已达79岁,部分地区如上海、深圳在84岁左右。预期寿命提高意味着平均退休后生存时间延长,养老金领取期相应拉长。对年金保险与养老产品而言,长寿化趋势不仅增加了领取期长度,也意味着死亡率水平与死亡率改善需要被纳入长期经营评估。在精算视角下,寿命改善将影响未来给付现值与准备金水平,从而对产品定价、风险管理与资产负债匹配提出更高要求。

长寿时代与少子化趋势叠加,形成“底部收缩”与“顶部加厚”同时发生的结构特征。养老保障体系因此面临双重压力:一方面,缴费人群增长放缓使得收入端承压;另一方面,领取期延长与老年人口规模增长使得支出端上升。由此,制度安排需要在资金筹集机制、待遇计发规则与多层次制度分工之间形成协调。

\subsubsection{老龄化社会与深度老龄化的判别标准}

老龄化社会通常以老年人口占比达到一定阈值为判别标准,即60岁以上老年人口占总人口的比例达到10\%,或65岁以上老年人口占总人口的比例达到7\%。在此标准下,我国于1999年进入老龄化社会。此后,60岁以上老年人口占比持续上升,2021年达到18.9\%,2024年达到22\%(约3.1亿)。

深度老龄化社会以65岁以上人口占总人口比例超过14\%为判别标准。相关数据表明,我国于2021年进入深度老龄化阶段(65岁以上人口占比14.2\%),2024年达到15.6\%。老龄化阶段的跃迁意味着养老保障制度从“覆盖扩张”逐步转向“支出增长与可持续性并重”,制度安排需要更强调长期财务平衡与多层次分担机制。

\begin{table}[htbp]
  \centering
  \caption{老龄化阶段的常用判别标准与我国阶段性节点。数据来源:演示文稿第10--12页。}
  \label{tab:aging_stage}
  \renewcommand{\arraystretch}{1.25}
  \begin{tabular}{p{0.2\textwidth}p{0.32\textwidth}p{0.34\textwidth}}
    \toprule
    阶段 & 判别标准 & 我国相关节点 \\
    \midrule
    老龄化社会 & 60岁及以上占比达到10\%或65岁及以上占比达到7\% & 1999年进入老龄化社会 \\
    深度老龄化 & 65岁及以上占比超过14\% & 2021年进入深度老龄化阶段 \\
    \bottomrule
  \end{tabular}
\end{table}

\subsubsection{复杂老龄化与结构性挑战}

老龄化在结构上表现为底部老龄化与顶部老龄化的叠加。底部老龄化主要体现为少儿端人口比重下降,其与社会发展与计划生育政策等多因素共同作用导致的少子化、独子化密切相关;顶部老龄化体现为老年人口绝对数量较多、预期寿命延长导致高龄化趋势增强。两者叠加使得人口结构呈现更为复杂的老龄化形态。

在复杂老龄化背景下,“未富先老”问题尤为突出。相关比较显示,部分国家进入深度老龄化时的人均国内生产总值水平较高,而我国在进入深度老龄化阶段时的人均国内生产总值水平相对较低。这一结构性差异意味着养老保障制度在财富积累基础、财政承受能力与市场化养老金融发展程度方面面临更为复杂的约束条件,从而要求在制度设计上更加注重多渠道财富储备与风险分担机制。

\subsubsection{老年人口抚养比与人口红利机制}

老年人口抚养比是衡量抚养压力的重要指标,其定义为每100名劳动年龄人口(15--64岁)需要负担的老年人口(65岁及以上)数量。抚养比越大,表示抚养负担越重;抚养比较低时,可为经济发展创造有利条件,通常被概括为“人口红利”阶段。相关数据表明,我国2021年65岁及以上老年人口抚养比已超过20\%。

抚养比上升意味着社会在养老金、医疗卫生与长期照护等方面的资源需求增加,同时也意味着劳动年龄人口对制度供给与家庭支持的分担能力下降。对基本养老保险而言,抚养比上升会提高现收现付制度在收入端与支出端的平衡难度;对居民养老规划而言,抚养比上升也提示单纯依赖家庭供养的风险增加,个体需要更早进行养老资金的长期准备,并在产品选择上考虑稳健积累与长期领取的匹配关系。

\subsection{养老需求的资金来源与结构特征}

\subsubsection{养老资金来源的调查结果与解释}

关于养老资金来源的调查结果显示,养老方式呈现多元化趋势。相关调研报告指出,传统家庭养老模式已不再完全适应现代社会需求,居民在养老保障方面存在一定的焦虑感并倾向于提前自我储备。在“未来将如何养老”的选项中,依靠自己的储蓄占比为67.4\%,依靠政府发放的基本养老保险金占比为12.9\%,延迟退休继续工作占比为10.4\%,依靠子女养老占比为9.3\%。该结果反映出居民对制度保障、家庭支持与自我储备的组合式安排的偏好正在形成,其中自我储备成为最主要的预期来源。

\begin{table}[htbp]
  \centering
  \caption{养老资金来源的调查结果汇总。数据来源:演示文稿第16页。}
  \label{tab:funding_source}
  \renewcommand{\arraystretch}{1.25}
  \begin{tabular}{p{0.46\textwidth}p{0.20\textwidth}}
    \toprule
    养老资金主要来源选项 & 占比 \\
    \midrule
    依靠自己的储蓄 & 67.4\% \\
    依靠政府发放的基本养老保险金 & 12.9\% \\
    延迟退休继续工作 & 10.4\% \\
    依靠子女养老 & 9.3\% \\
    \bottomrule
  \end{tabular}
\end{table}

从需求结构看,上述结果揭示了三方面含义。第一,养老资金准备呈现明显的个人化特征,居民更倾向于将养老视为跨期储蓄与投资的长期规划问题。第二,制度养老仍具有基础性地位,但其更多被视为“保基本”的底层保障,难以覆盖全部养老开支。第三,延迟退休与继续工作被视为重要补充渠道,反映出劳动供给与养老资金准备之间的互动关系。

\subsubsection{养老风险类型与养老金融工具匹配}

养老需求并不仅仅体现为退休后收入缺口的弥补,还涉及多维风险的综合管理。一方面,长寿风险使得退休期长度具有不确定性,若退休后生存期超出预期,则养老资金可能面临持续供给压力;另一方面,健康风险与长期照护需求可能导致退休后支出结构发生显著变化,尤其在高龄阶段更为明显。除此之外,投资收益波动与缴费中断等因素也会影响养老资金积累路径。由此,养老金融工具的选择需要兼顾稳健积累、风险保障与领取安排等要素,并在制度安排与市场化产品之间形成互补。

\subsubsection{人口结构变化趋势与养老需求扩张}

人口结构变化趋势的长期性决定了养老需求扩张具有持续性。相关预测图示表明,我国60岁以上老年人口数量及占比将持续上升,并在较长时期内维持较高水平。在结构表述上,2020年大致呈现“每5个人中有1个老人”的格局,而到2050年将可能呈现“每3个人中有1个老人”的格局。人口结构的这一变化意味着养老保障体系需要在更大规模的老年人口背景下运行,养老金融市场也将面临更广泛的需求基础。

\begin{figure}[htbp]
  \centering
  \includegraphics[width=0.88\textwidth]{figure/PPT_3_10.png}
  \caption{60岁以上老年人口数量及占比变化趋势示意。图像来源:演示文稿第14页。}
  \label{fig:aging_trend}
\end{figure}

养老需求扩张不仅体现在老年人口规模增加,也体现在养老需求的期限结构与产品偏好变化。随着退休期延长,养老资金的跨期规划更强调长期稳健积累与长期领取安排;同时,在多层次养老保障体系逐步健全的背景下,养老需求也将更多体现为制度保障与市场化产品的组合使用,从而提升商业养老保险、商业保险年金与个人养老金产品的需求空间。

\section{国家应对老龄化的制度安排与行业政策}

本节围绕国家积极应对人口老龄化的总体部署与保险业政策导向进行阐释。中长期规划强调人口老龄化的基本国情与挑战机遇并存的判断,提出通过夯实财富储备与提高社会保障能力的双重路径构建应对体系,并将多层次养老保险制度建设置于关键位置。保险业高质量发展政策强调提升民生保障服务能力,提出积极发展第三支柱养老保险、发展商业保险年金与推动专属商业养老保险等具体方向,为商业养老保险的产品创新与经营发展提供了明确的制度环境。

\subsection{国家积极应对人口老龄化的总体部署}

《国家积极应对人口老龄化中长期规划》明确指出,人口老龄化是社会发展的重要趋势,是人类文明进步的体现,也是今后较长一段时期我国的基本国情。人口老龄化对经济运行全领域、社会建设各环节、社会文化多方面乃至国家综合实力和国际竞争力具有深远影响,挑战与机遇并存。该判断意味着应对老龄化不应被视为单一社会保障问题,而应在经济发展、公共服务供给、社会治理与产业发展等多个维度形成系统性安排。

规划在政策框架上强调财富储备与社会保障能力的协同提升。一方面,要增强应对人口老龄化的经济基础,通过保持经济持续稳定增长、优化经济发展结构、提高经济发展质量效益,促进经济发展与人口老龄化进程相适应;同时,通过完善国民收入分配体系、加大财政支持力度、促进企业财富积累与合理分配,鼓励家庭、个人建立养老财富储备,稳步增加全社会的养老财富储备。另一方面,要注重提高社会保障能力,加快建立覆盖全民、城乡统筹、权责清晰、保障适度、可持续的多层次养老保险制度,并健全医疗保障、长期照护保障以及社会福利和社会救助体系。

社会财富储备强调应对老龄化的资源基础,其本质在于通过经济增长与合理分配形成可用于养老保障与相关公共服务的资源池。在制度层面,财富储备并不等同于单一主体的积累,而是包含国家层面的财政资源、企业层面的长期资金积累以及家庭与个人层面的养老财富准备。多层次养老保险制度建设则是将上述资源在制度化框架下进行分层配置,通过不同支柱之间的功能分工实现风险分担与保障供给的结构优化。第一支柱强调保基本与普惠覆盖,第二支柱强调单位与个人共同参与的补充保障,第三支柱强调个人自主积累与市场化产品供给。多层次制度的提出与完善,正是对财富储备逻辑的制度化表达。

老龄化带来的制度压力不仅体现在养老金支付,也体现在医疗保障与长期照护需求上。规划强调健全老有所医的医疗保障制度,并建立多层次长期照护保障制度,实施兜底性长期照护服务保障行动计划。在养老保障体系的分析框架中,养老金制度与医疗保障、长期照护制度之间具有高度互补性:养老金主要应对退休后收入下降导致的消费能力下降风险,而医疗与照护制度主要应对健康支出波动与失能风险。三者协同有助于提升老年阶段的整体保障水平,并减少单一制度承受过度压力的风险。

\subsection{保险业服务民生保障的政策导向}

《关于加强监管防范风险推动保险业高质量发展的若干意见》提出提升保险业服务民生保障水平的要求,强调积极发展第三支柱养老保险,大力发展商业保险年金,以满足人民群众多样化养老保障和跨期财务规划需求,并鼓励开发适应个人养老金制度的新产品和专属产品,支持养老保险公司开展商业养老金业务,推动专属商业养老保险发展。同时,政策强调丰富与银发经济相适应的保险产品、服务和保险资金支持方式,依法合规促进保险业与养老服务业协同发展。

上述政策导向体现出监管部门对保险业在养老保障体系中的功能定位:保险业不仅是风险保障供给者,也是长期资金管理者与跨期财务规划服务提供者。商业保险年金与专属商业养老保险等产品形态之所以受到重视,原因在于其能够通过长期合同安排、稳健积累机制与领取规则设计,为居民提供与养老周期相匹配的金融解决方案。

银发经济强调与老龄人口相关的产品供给与服务体系建设。保险业与银发经济的联动主要体现在两方面。其一,养老保险产品与养老服务、照护服务之间具有潜在的衔接空间,在风险有效隔离前提下,通过服务整合可提升养老解决方案的整体性与可获得性。其二,保险资金具有期限长、规模大、稳定性较强等特征,能够在依法合规与风险约束下为养老服务体系建设提供支持方式,从而提升养老服务供给能力。政策提出丰富保险资金支持方式,旨在引导保险业在风险可控前提下发挥长期资金优势,更好服务民生保障与养老产业发展。

\section{三支柱养老保障体系的结构与运行机制}

三支柱养老保障体系通过分层分担机制构成多渠道养老供给结构。第一支柱覆盖面广、资金规模大,强调保基本并通过统筹互济与名义账户记录相结合的方式运行,但其可持续性对人口结构变化高度敏感。第二支柱以个人账户与市场化投资为特征,企业年金与职业年金在缴费机制、参与激励与覆盖范围上存在差异,覆盖面与参与度仍有提升空间。第三支柱个人养老金制度通过账户完全积累与税收优惠机制促进长期养老财富准备,但在实际缴费转化与规模扩张方面仍处于发展初期。上述制度结构为商业养老保险与养老金融市场的发展提供了制度空间与需求基础。

\subsection{三支柱框架的功能分工}

三支柱养老保障体系以分层分担为基本逻辑,通过制度安排将养老保障供给划分为不同层次,从而在普惠性、补充性与个体自主性之间形成结构化组合。第一支柱以基本养老保险为核心,覆盖范围广,强调基本生活保障;第二支柱以企业年金与职业年金为代表,强调在基本养老保险基础上的补充保障,并以单位与个人共同缴费、个人账户管理为主要特征;第三支柱以个人养老金为代表,强调个人自主缴费、账户完全积累与税收优惠等制度安排,并允许通过多类金融产品实现长期资金积累与养老领取。

在精算分析中,三支柱框架的意义不仅在于制度分类,更在于风险分担与资金来源结构的优化。第一支柱通常采用现收现付为主的筹资逻辑,依赖当期缴费支持当期给付;第二与第三支柱更强调完全积累与个人账户机制,更有利于在个体层面实现资产积累与跨期配置。三者共同构成养老保障供给的组合体系,能够在覆盖面、保障水平与可持续性之间形成更为稳健的结构。

\begin{table}[htbp]
  \centering
  \caption{三支柱养老保障体系的功能分工与运行机制比较。数据来源:演示文稿第21--32页。}
  \label{tab:pillars_compare}
  \renewcommand{\arraystretch}{1.2}
  \begin{tabular}{p{0.13\textwidth}p{0.25\textwidth}p{0.25\textwidth}p{0.27\textwidth}}
    \toprule
    支柱 & 保障定位 & 资金筹集与账户特征 & 运行要点 \\
    \midrule
    第一支柱 & 基础保障与普惠覆盖 & 多方负担;统筹与个人账户相结合;个人账户为名义账户制 & 覆盖面大,受人口结构影响显著,强调制度可持续性 \\
    第二支柱 & 补充保障 & 单位与个人共同缴费;个人账户制;市场化投资运营 & 与单位治理与激励相关,覆盖面有待提升 \\
    第三支柱 & 自主积累与税收激励 & 个人完全缴费;个人账户制与资金账户唯一性;完全积累 & 允许多类金融产品配置,强调长期保值与领取规则 \\
    \bottomrule
  \end{tabular}
\end{table}

\subsection{第一支柱基本养老保险}

\subsubsection{参保范围与筹资机制}

基本养老保险强调全民参保、应保尽保。其覆盖对象包括企业、机关事业单位、社会团体的在职及退休职工,以及未就业的城乡居民。资金筹集具有多元性,由国家、单位与个人共同负担。一般而言,单位缴费比例为本人工资的16\%,个人缴费比例为本人工资的8\%,国家通过调剂机制与对困难群体的补贴机制参与制度运行。

基本养老保险实行社会统筹与个人账户相结合。用人单位缴纳的金额进入统筹基金,用于支付当前退休人员的养老金;个人缴纳的金额计入个人账户,并采用“名义账户制”,资金与统筹账户合并管理。这一结构在制度上兼顾了统筹互济与个体权益记录两方面功能:统筹基金体现再分配与风险共担,个人账户体现权益记录与待遇计发的个体化基础。

\subsubsection{保障水平与规模特征}

基本养老保险以保障退休人员基本生活为目标,其替代率水平通常为40\%--50\%,并存在发达国家60\%--70\%的对照参照。规模方面,截至2024年底,参保人数约10.7亿人,2024年收入约8.2万亿元,支出约7.3万亿元,累计结余约8.7万亿元。上述规模表明第一支柱在覆盖面与资金规模上具有基础性地位,同时也意味着其财务可持续性对宏观经济与人口结构高度敏感。

替代率作为衡量养老金保障水平的常用指标,其含义在于退休后养老金收入与退休前工资收入之间的比例关系。替代率水平与居民退休后消费维持能力密切相关。若仅依赖第一支柱而替代率水平偏低,则居民在退休后可能面临消费水平显著下降的风险;在此情形下,第二支柱与第三支柱的补充作用将更加突出。替代率的比较也提示,不同制度安排在保障水平与可持续性之间存在权衡,保障水平提升通常意味着更高的筹资强度或更大的财政支持需求,因此多层次结构有助于在普惠性与差异化需求之间形成更为合理的分工。

\begin{table}[htbp]
  \centering
  \caption{第一支柱基本养老保险的运行要点概览。数据来源:演示文稿第21页。}
  \label{tab:pillar1_overview}
  \renewcommand{\arraystretch}{1.25}
  \begin{tabular}{p{0.22\textwidth}p{0.70\textwidth}}
    \toprule
    维度 & 主要内容 \\
    \midrule
    覆盖对象 & 企业、机关事业单位、社会团体在职及退休职工;未就业城乡居民 \\
    筹资结构 & 单位16\%,个人8\%,国家调剂与补贴机制 \\
    制度结构 & 社会统筹与个人账户相结合;个人账户为名义账户制 \\
    保障定位 & 保基本,替代率约40\%--50\% \\
    规模特征 & 参保人数约10.7亿;2024年收入约8.2万亿、支出约7.3万亿、累计结余约8.7万亿 \\
    \bottomrule
  \end{tabular}
\end{table}

\subsubsection{可持续性挑战与制度机理}

关于基本养老保险的可持续性问题,相关测算表明支付压力可能不断提升,现有制度下在2035年存在透支风险。其机制解释可从人口结构与制度安排两方面理解。人口层面,老龄化使得基金收入减少、支出增加,抚养比上升导致现收现付制度的平衡难度加大。制度层面,现收现付制度依赖当期缴费支持当期给付,若缴费人群增速放缓而领取人群规模上升,则收支缺口可能扩大。与此同时,精算不平衡也会影响制度压力,例如60岁退休人员的发放月数为139个月,但实际平均余命可能大于20年,领取期延长将提高实际给付负担。

此外,再分配特性会影响个体缴费积极性。统筹互济机制体现社会公平与风险共担,但若个体感受到缴费与待遇之间的对应关系不足,可能导致缴费激励下降。在制度运行中,需要在互济性与激励性之间保持合理平衡,并通过完善制度参数、优化调剂机制与提升制度透明度等方式增强制度可持续性与公众预期稳定性。

\subsection{第二支柱企业年金与职业年金}

\subsubsection{企业年金的制度特征与缴费规则}

企业年金是在依法参加基本养老保险基础上,由企业及其职工自主建立的补充养老保险制度。其特点主要表现为自愿性、补充性与专属性。自愿性意味着企业自主决定是否建立,职工也可自愿参加;补充性意味着其定位为基本养老保险的补充;专属性意味着资金归职工个人所有,通常在退休或满足特定条件时方可领取。

企业年金的缴费机制由企业缴费与个人缴费共同构成,缴费比例由企业与职工协商确定但需符合国家规定:企业缴费不超过职工工资总额的8\%,企业与个人合计缴费不超过职工工资总额的12\%,个人缴费比例通常为1\%--4\%。企业年金实行个人账户制,记录缴费、投资收益与待遇支付等信息;基金运作采取市场化投资运营,由专业机构管理。待遇支付条件包括退休、完全丧失劳动能力、出国定居与身故等情形。

\subsubsection{税收优惠与企业治理含义}

企业年金具有明确的税收优惠安排。企业缴费部分可在企业税前扣除;个人缴费部分暂不征收个人所得税,领取时按“工资薪金所得”单独计税;投资收益暂不征收所得税,领取时合并计税。税收优惠在制度上降低了建立与参与企业年金的成本,并在一定程度上提升了企业参与的激励。企业年金对企业治理的意义体现在吸引与留住人才、提升雇主形象以及提供多样化投资风格选择等方面,同时也对企业财务管理、薪酬结构与长期激励机制提出了相应要求。

\subsubsection{覆盖现状与发展空间}

企业年金覆盖人数呈现放缓趋势。相关数据显示,2024年底全国约1300万家企业中,仅约16万家建立企业年金制度;参加职工数约3242万,覆盖率约7\%。同年管理资产规模约3.6万亿元,同比增长14.3\%,当年投资收益率约4.8\%。上述事实表明,企业年金在资产规模上增长较快,但在企业覆盖面与参保覆盖率方面仍存在较大提升空间。

\begin{table}[htbp]
  \centering
  \caption{企业年金制度要点与运行数据概览。数据来源:演示文稿第23--25页。}
  \label{tab:enterprise_annuity}
  \renewcommand{\arraystretch}{1.25}
  \begin{tabular}{p{0.22\textwidth}p{0.70\textwidth}}
    \toprule
    维度 & 主要内容 \\
    \midrule
    制度定义 & 企业及职工在参加基本养老保险基础上自主建立的补充养老保险制度 \\
    关键特征 & 自愿性、补充性、专属性;资金归职工个人所有 \\
    缴费约束 & 企业缴费不超过8\%;企业与个人合计不超过12\%;个人通常1\%--4\% \\
    账户与投资 & 个人账户制;市场化投资运营,由专业机构管理 \\
    运行数据 & 2024年底16万家企业建立;参保职工约3242万,覆盖率约7\%;资产规模约3.6万亿,收益率约4.8\% \\
    \bottomrule
  \end{tabular}
\end{table}

\subsubsection{职业年金的强制性与制度安排}

职业年金是机关事业单位及其编制内工作人员在依法参加基本养老保险基础上建立的补充养老保险制度。其特点表现为强制性、补充性与专属性。强制性意味着职业年金是机关事业单位的法定责任,缴费比例由国家统一规定。具体而言,单位按职工工资总额的8\%缴纳并直接计入职工个人账户,个人按本人缴费工资的4\%缴纳并由单位代扣后计入个人账户。职业年金同样实行个人账户制并采取市场化投资运营。待遇支付条件包括退休、完全丧失劳动能力、出国定居与身故等。相关数据显示,职业年金已实现全面覆盖;截至2024年底,管理资产规模约3.1万亿元,自2019年市场化运作以来年平均投资收益率约4.4\%。

\subsection{第三支柱个人养老金}

\subsubsection{制度演进与试点推进}

第三支柱围绕个人自主积累与税收优惠机制逐步推进。制度演进上,2018年4月启动个人税收递延型商业养老保险试点;2021年5月推进专属商业养老保险试点;2022年4月推动个人养老金发展并形成制度框架;2024年12月个人养老金制度在全国范围内正式实施。制度推进过程中,个人养老金业务逐步覆盖保险、银行、理财、基金等金融领域,并推广至全国。

\begin{figure}[htbp]
  \centering
  \includegraphics[width=0.86\textwidth]{figure/PPT_3_27.png}
  \caption{第三支柱相关试点与制度推进的时间序列示意。图像来源:演示文稿第27页。}
  \label{fig:pillar3_timeline}
\end{figure}

\subsubsection{制度框架与政策体系}

个人养老金制度框架在2022年全面落地,国务院办公厅发布《关于推动个人养老金发展的意见》,并配套《个人养老金实施办法》。税收优惠政策由财政、税务部门发布并明确具体政策安排,公募基金、商业银行与理财公司、保险机构等分别在监管框架下制定个人养老金业务管理规定。制度框架的核心在于以个人账户制为基础,通过业务信息平台实现账户管理与信息交互,并由符合条件的商业银行开办个人养老金资金账户,用于资金缴存与养老金支付等。资金账户具有唯一性,参加人只能选择一家符合条件的商业银行确定一个资金账户。

\subsubsection{税收优惠机制与领取税制}

个人养老金实施递延纳税优惠政策。具体而言,个人向个人养老金资金账户的缴费,按照12000元/年的限额标准,在综合所得或经营所得中据实扣除;计入个人养老金资金账户的投资收益暂不征收个人所得税;个人领取的个人养老金不并入综合所得,单独按照3\%税率计算缴纳个人所得税。上述税制安排在缴费阶段体现税前扣除,在积累阶段体现投资收益免税,在领取阶段体现单独计税,从而形成鼓励长期积累的制度激励。

\subsubsection{实施模式与业务流程}

个人养老金制度明确参加人范围为中国境内参加城镇职工基本养老保险或城乡居民基本养老保险的劳动者。制度运行需要依托金融监管部门组织建设的业务信息平台以及参与金融机构。业务流程上,参加人首先开立个人养老金账户,继而在商业银行开立个人养老金资金账户,用于资金缴存、养老金支付等;随后,参加人可在资金账户内自主选择购买符合规定的储蓄存款、理财产品、商业养老保险、公募基金等金融产品。缴费限额为1.2万元/年,缴费方式可按月、分次或按年进行。领取规定上,达到退休年龄可以领取;此外,在完全丧失劳动能力、出国(境)定居或符合国家规定的其他情形下也可领取;领取方式包括分次领取与一次性领取。制度强调完全积累,即缴费完全由参加人个人承担,资金在账户内长期积累并按规定享受税收优惠。

\subsubsection{可投资产品类别与产品特征要求}

个人养老金可投资的产品包括储蓄存款、理财产品、商业养老保险、公募基金等,产品类别较为广泛。制度对个人养老金产品的基本特征提出要求,即运作安全、成熟稳定、标的规范并侧重长期保值。该要求反映出个人养老金制度以长期养老目标为核心,强调风险可控与长期稳健积累,避免短期波动对养老目标产生过度影响。

\subsubsection{业务现状与结构性特征}

截至2025年上半年末,个人养老金业务规模仍较小,但发展空间较大。制度覆盖人数较多而实际参加人数相对较少;开户人数多而缴费人数少;低缴费人数多而高缴费人数少;产品数量多而购买产品人少。具体数据显示,个人养老金账户开户人数超过7000万,缴纳金额的账户数约2100万,占比约30\%;业务规模约500亿元;平均每个账户缴费约2000元。上述结构性特征提示,个人养老金制度落地后仍处于培育阶段,未来的发展重点可能包括提升缴费转化率、提高缴费水平与增强产品与服务的可获得性与适配性。

个人养老金业务的上述结构性特征,反映出制度从“可开户”向“稳定缴费与长期持有”的转化仍需要时间。其原因既可能与居民对税收优惠规则与领取规则的理解程度有关,也可能与可投资产品的期限结构、风险特征与收益呈现方式有关。由于个人养老金强调长期保值与养老属性,短期内缴费与购买行为往往受到现金流约束与风险偏好影响。制度培育阶段的核心任务之一,是在保持制度稳健与风险可控的前提下,通过产品供给优化与服务流程完善,提升账户缴费的持续性与养老金产品配置的可及性,从而使第三支柱在多层次养老保障体系中逐步发挥更为稳定的补充作用。

\begin{table}[htbp]
  \centering
  \caption{个人养老金业务关键指标概览。数据来源:演示文稿第34页。}
  \label{tab:pension_status}
  \renewcommand{\arraystretch}{1.25}
  \begin{tabular}{p{0.34\textwidth}p{0.36\textwidth}}
    \toprule
    指标 & 数值 \\
    \midrule
    账户开户人数 & 7000万以上 \\
    缴费账户数 & 2100万左右(约30\%) \\
    业务规模 & 约500亿元 \\
    平均账户缴费 & 约2000元 \\
    \bottomrule
  \end{tabular}
\end{table}

\section{商业保险年金与专属商业养老保险}

商业保险年金与专属商业养老保险是商业养老保险体系的重要组成部分。商业保险年金在政策导向下强调养老风险管理与长期资金稳健积累功能,并在规模与产品创新上具有显著发展空间。专属商业养老保险通过双账户管理、保证利率机制与可转换安排,结合灵活缴费与长期领取约束,为居民提供与养老周期相匹配的跨期财务规划工具。其经营与监管要求强调长期销售激励、风险管控与长期投资考核机制,体现养老产品“长期合同、长期资金、长期责任”的经营逻辑。

\subsection{商业保险年金的发展导向与概念界定}

关于大力发展商业保险年金的监管通知提出,以政策发布为契机推动保险公司大力发展商业保险年金,不断优化产品与服务,打造群众信赖的行业品牌。政策强调引导保险公司发挥精算技术、长期产品开发与长期资金管理优势,为居民提供丰富多样的养老保障与跨期财务规划服务,并提出优化个人养老金产品供给、提升产品多样性与投保便利度,支持创设兼具养老风险保障与财富管理功能的新型产品,建立健全统计制度、加强监管制度体系建设、维护消费者权益等要求。相关资料显示,截至2024年1月商业保险年金规模已超过6万亿元。

商业保险年金在概念上指商业保险公司开发的具有养老风险管理、长期资金稳健积累等功能的产品,包括符合条件的年金保险、两全保险、商业养老金等。该概念强调养老风险管理与长期资金积累两项核心功能,并以产品形态的包容性涵盖不同类型的寿险与养老险产品。对精算经营而言,商业保险年金的长期属性要求在定价、准备金评估与资产负债管理中充分考虑期限匹配与现金流稳定性。

\subsection{专属商业养老保险的产品结构与机制}

\subsubsection{产品示例的主要条款结构}

专属商业养老保险作为具有“双账户、有保底、可转换”等特征的产品形态,在条款结构上通常包含承保规则、保障责任、账户管理与退保规则等关键要素。以产品示例为例,承保规则包括投保年龄为0--80岁,初始费用不超过5\%。保障责任方面,身故金在积累期给付标准为账户价值与已交保费的较大者;在领取期,终身给付方式为账户价值减已领取年金与0之间的较大者,固定期限给付方式为未领取年金与账户价值减已领取年金之间的较大者。养老金领取开始年龄通常不低于法定退休年龄或60岁;养老金领取期限可选择10年、15年、20年或终身;是否在投保时锁定领取标准可设置为否。账户管理方面,产品设置稳健账户与进取账户的保证利率水平,并可按年结方式结息;退保规则在积累期按年度设定退保比例或按账户价值与已交保费之间差额的比例计算,领取期退保金额为0。

\begin{table}[htbp]
  \centering
  \caption{专属商业养老保险产品示例的关键条款要点。数据来源:演示文稿第36页。}
  \label{tab:exclusive_example}
  \renewcommand{\arraystretch}{1.25}
  \begin{tabular}{p{0.24\textwidth}p{0.68\textwidth}}
    \toprule
    条款维度 & 主要内容 \\
    \midrule
    公司与产品 & 新华养老,盈佳人生C款 \\
    投保年龄 & 0--80岁 \\
    初始费用 & 不超过5\% \\
    身故金积累期 & $\max(\text{账户价值},\ \text{已交保费})$ \\
    身故金领取期 & 终身给付:$\max(\text{账户价值}-\text{已领取年金},\ 0)$;固定期限:$\max(\text{未领取年金},\ \text{账户价值}-\text{已领取年金})$ \\
    养老金起领年龄 & 不低于法定退休年龄或60岁 \\
    养老金领取期限 & 10年、15年、20年或终身 \\
    账户保证利率 & 稳健账户1.75\%,进取账户0.5\%;每年可转换一次且不收取费用 \\
    结息方式 & 年结 \\
    退保规则 & 1--5年:已交保费的95\%/97\%/99\%/100\%/100\%;6--10年:已交保费的100\%$+(\text{账户价值}-\text{已交保费})\times 75\%$;11年及以后:已交保费的100\%$+(\text{账户价值}-\text{已交保费})\times 90\%$;领取期:0 \\
    \bottomrule
  \end{tabular}
\end{table}

\subsubsection{双账户管理与缴费灵活性}

专属商业养老保险通常提供稳健型与进取型两个账户,采用不同的保证利率与投资方案,便于客户根据自身风险偏好进行选择,并允许在账户之间进行转换。该“双账户、有保底、可转换”的业务模式具有鲜明特征:稳健账户强调保证与稳健积累,进取账户强调在风险可控前提下追求相对更高的长期收益空间;账户可转换机制为个体在不同生命周期阶段调整风险偏好提供便利。与此相对应,专属商业养老保险在缴费模式上具有较强灵活性,类似万能保险与投资连结保险,允许不定期缴费,便于客户在积累期根据资金状况灵活安排缴费节奏。

\subsubsection{最低保证利率与结算利率的比较}

专属商业养老保险在利率机制上通常同时涉及最低保证利率与年度结算利率。最低保证利率体现产品在账户层面对资金积累的底线承诺;年度结算利率反映保险公司在一定经营与投资结果基础上确定的当期结算水平。相关产品比较显示,不同公司与产品在稳健型与进取型账户的最低保证利率与年度结算利率上存在差异。例如,部分产品稳健型最低保证利率为2.00\%,进取型最低保证利率为0.00\%或0.50\%;2024年度结算利率在稳健型账户上约为2.85\%--4.07\%,在进取型账户上约为3.00\%--4.12\%。

\begin{table}[htbp]
  \centering
  \caption{部分专属商业养老保险产品的保证利率与2024年度结算利率对比。数据来源:演示文稿第38页。}
  \label{tab:settlement_rate}
  \renewcommand{\arraystretch}{1.15}
  \resizebox{\textwidth}{!}{
  \begin{tabular}{p{0.14\textwidth}p{0.16\textwidth}ccccc}
    \toprule
    公司简称 & 产品名称 & 稳健型最低保证利率 & 进取型最低保证利率 & 2024稳健型结算利率 & 2024进取型结算利率 \\
    \midrule
    中国人寿 & 鑫享宝专属商业养老保险 & 2.00\% & 0.00\% & 3.10\% & 3.30\% \\
    太保寿险 & 易生福专属商业养老保险A款 & 2.00\% & 0.50\% & 2.90\% & 3.10\% \\
    人保寿险 & 福寿年年专属商业养老保险F款 & 2.00\% & 0.50\% & 3.40\% & 3.10\% \\
    泰康人寿 & 臻享百岁专属商业养老保险 & 2.85\% & 0.50\% & 2.85\% & 3.00\% \\
    太平人寿 & 岁岁金生专属商业养老保险 & 2.00\% & 0.00\% & 2.90\% & 3.30\% \\
    国民养老 & 共同富裕B专属商业养老保险 & 2.00\% & 0.00\% & 4.07\% & 4.12\% \\
    \bottomrule
  \end{tabular}}
\end{table}

利率机制的存在使得专属商业养老保险能够在长期合同框架下兼顾保证与浮动收益。最低保证利率为长期积累提供底线保障,结算利率则通过年度机制反映经营结果并在一定程度上实现收益分配。对投保人而言,该机制将养老资金积累与长期投资收益联系起来;对保险公司而言,该机制要求在保证成本控制、投资收益提升与资产负债匹配之间形成协调,从而避免利率风险与期限错配风险对长期经营造成不利影响。

\subsubsection{产品供给、领取约束与配套机制}

专属商业养老保险的制度安排强调消费者达到60周岁及以上方可领取养老金,且领取期限不短于10年。该领取约束体现养老属性与长期保障导向,有助于将资金锁定于养老目标并强化长期领取特征。

退保规则在专属商业养老保险中具有重要的行为约束作用。积累期退保金额通常与已交保费、账户价值及其差额相关,并在不同年度设置不同的退保比例。以示例产品为例,前五年退保金额与已交保费按95\%至100\%的比例计算,体现早期退保的成本约束;在第六年至第十年以及第十一年及以后,退保金额在返还已交保费基础上,对账户价值超过已交保费的部分按一定比例返还,体现对长期持有的激励。领取期退保金额为0的安排进一步强化了养老金领取阶段资金的养老属性,强调资金在领取期用于持续给付而非短期退出。这一规则结构有助于降低逆向选择与短期套利行为对长期经营的不利影响,同时也要求投保人在进入领取期前充分评估自身流动性需求。

在经营管理方面,相关要求强调探索建立长期发展的内部管理机制,包括长期销售激励考核机制、风险管控机制与较长期限的投资考核机制等,以适应养老险长期经营的特点。同时,专属商业养老保险强调服务新市民与灵活就业人员养老需求,并允许相关单位以适当方式提供缴费支持;在风险有效隔离前提下,鼓励将业务发展与养老、照护服务等相衔接,以满足差异化养老需求。

\section{年金产品的定价逻辑与盈利性管理}


年金产品定价以预定死亡率、预定费用率与预定利率为核心假设,并以精算现值平衡关系构成费率厘定的基础框架。在此基础上,利润测试与新业务价值评估通过现实假设衡量业务可分配盈余与资本成本,形成对费率合理性的经营性检验。首日亏损率、偿付能力压力与敏感性测试等工具用于刻画长期业务在短期财务与资本约束下的可承受性。三差平衡框架从利差、死差与费差三维度提出管理要点,强调保证成本控制、风险评估更新与运营效率提升的协同,从而支撑养老年金产品的长期稳健经营。

\subsection{定价要素与精算平衡关系}

\subsubsection{预定假设与基本定价框架}

年金产品定价通常基于预定死亡率、预定费用率与预定利率等关键假设。预定死亡率用于刻画被保险人在不同年龄的生存与死亡概率结构,是年金给付与死亡给付现值计算的基础;预定费用率用于反映产品取得与维持过程中预期发生的费用水平;预定利率用于将未来现金流折现至评估时点,是长期产品定价的重要参数。在监管框架下,预定利率通常受到上限约束,例如传统保险产品的预定利率最高水平存在明确规定。预定利率作为折现假设并不等同于实际投资收益水平,而是定价环节为确保稳健性而采用的制度化假设。

在上述假设基础上,年金产品的基本精算平衡关系可表述为:每期保费的精算现值之和等于每期年金给付的精算现值之和、死亡给付的精算现值之和以及预定附加费用的精算现值之和之和。该平衡关系是费率厘定的核心逻辑,体现了在预定假设下保费收入应能够覆盖预期给付与费用支出。

\begin{equation}
\sum_{t} \text{保费}_{t}\,v^{t}
=
\sum_{t} \text{年金给付}_{t}\,v^{t}
+
\sum_{t} \text{死亡给付}_{t}\,v^{t}
+
\sum_{t} \text{附加费用}_{t}\,v^{t},
\end{equation}

其中,$v^{t}$表示按预定利率折现的贴现因子,求和范围由缴费期、领取期与保障责任期限共同决定。该公式强调在定价层面的“收支相等”,其目的在于保证费率在预定假设下的充足性与稳健性。

\subsubsection{利润测试与新业务价值率}

在定价平衡关系之上,产品还需要通过利润测试来证明费率合理,即业务承担的风险与创造的价值相适应。利润测试强调在更为现实的经营假设下,对未来各期利润进行测算并折现至当前时点,从而评估新业务价值水平。新业务价值率可表述为未来各期可分配盈余现值之和与首年保费之比。可分配盈余的计算强调经营层面的收入与支出结构,通常包括保费收入与投资收益等收入项,扣除赔付支出、费用支出、退保支出、准备金提转差、税费以及资本成本等支出项。资本成本反映开展业务需要满足监管偿付能力要求而占用资本的机会成本。由此,利润测试与价值评估在本质上体现为以现实假设衡量业务的经济价值,而非仅以预定假设形成的费率平衡关系作为评价依据。

在价值评估中,投资收益假设往往采用更贴近长期资产配置能力的水平。例如,在预定利率受限的定价框架下,折现可采用2\%等稳健假设以保证费率充足性,但企业在价值测算时可能基于长期资金运用能力采用更高的长期投资收益假设,例如4\%。同样地,死亡率假设也需要考虑随医疗卫生条件改善而出现的死亡率改善趋势;对于年金产品而言,死亡率改善会延长预期领取期并提高给付现值,从而影响可分配盈余与价值率。费用假设则需要反映真实发生的佣金、手续费、职场与人力成本以及其他业务费用,并结合退保行为与准备金提转差等因素形成对利润的完整刻画。由此,价值评估的关键在于以现实假设将长期经营结果量化,并据此检验产品在风险承担与价值创造之间的匹配性。

\subsubsection{首日亏损率与经营约束}

在会计与监管框架下,首日亏损需要在保单签发时点确认,首日利得需要在保险期间摊销。首日亏损率用于刻画新单在签发时点的损益特征,其高低与产品费用结构、佣金水平、准备金计提规则以及定价假设等因素相关。首日亏损过高会对短期利润与偿付能力形成压力,而首日利得的摊销则需要与长期责任匹配。因而,在长期产品经营中,首日损益管理与长期价值创造之间需要形成一致性安排,避免通过短期财务表现牺牲长期风险控制或反之。

\subsubsection{偿付能力压力与敏感性测试}

年金产品具有长期责任属性,资产负债期限错配、利率波动与死亡率改善等因素可能通过准备金与资本占用等路径影响偿付能力水平。产品在开发与定价阶段通常需要评估偿付能力压力,并开展敏感性测试,以识别关键假设变动对利润与资本占用的影响。例如,投资收益率下降、费用水平上升或死亡率改善速度快于预期均可能降低可分配盈余并增加资本压力。敏感性测试的目的在于在产品上市前识别主要风险敞口,为后续风险控制与资产负债管理提供依据。

\subsection{三差平衡框架与经营管理}

利差管理的核心在于在保证成本与投资收益之间形成可持续的收益空间。管理措施包括降低新业务保证成本并鼓励开发浮动收益型产品,通过产品结构设计降低刚性保证负担;同时,通过多举措提升投资收益率,提升长期资金运用效率。在资产负债联动方面,需要实现成本收益匹配、期限匹配与现金流匹配,确保资产端收益与负债端成本在长期维度上协调,从而控制利率风险与再投资风险。

死差管理强调对生存与死亡风险的准确评估,确保定价充足并减少死亡率偏差对经营结果的不利影响。管理措施包括精准评估风险、丰富产品多样性以及开展生存调查等。对于年金类产品而言,长寿风险尤为关键,死亡率改善将延长领取期并提高给付现值,因而需要在经验数据积累与假设更新机制上保持敏感性。死差管理不仅关系到定价与准备金水平,也关系到风险选择与产品组合管理。

费差管理关注实际费用与预定费用之间的差异,其核心在于通过成本控制与规模效应提升经营效率。管理措施包括资源聚焦绩优队伍以优化变动成本,落实降本增效与数字化运营以降低固定成本,并通过业务规模扩大形成规模效应。长期产品经营中,费用的前置性特征往往使得首年费用压力较大,因而费差管理与首日亏损管理具有内在关联。通过优化渠道结构、提升运营效率与强化精细化管理,可以在一定程度上改善费用结构并提升长期价值率。

\section{养老方式变迁与养老资金准备示例}

养老方式变迁反映出家庭结构变化与人口结构变化对代际支持机制的影响,制度养老与自我储备的重要性随之上升。养老资金准备可以通过数量化演算建立直观框架,本小节通过示例表明退休后支出水平与领取期长度对资金需求规模具有决定性影响。该框架为理解个人养老金、商业保险年金与专属商业养老保险等工具的跨期规划属性提供了直观基础。

随着人口结构与家庭结构变化,家庭养老的不确定性日益凸显。子女养老在传统模式中承担重要角色,但在少子化、独子化与流动性增强背景下,子女支持的可得性与稳定性存在不确定性。相较之下,制度养老与市场化养老金融工具更强调规则明确、权益可记录与跨期规划可执行。养老方式的变迁可以通过对比性示例加以说明:在长期抚养投入相近的条件下,子女未来是否能够持续提供稳定的养老支持并不具有确定性;而制度安排与金融合同在一定程度上能够提供更为可预期的现金流安排。该对比强调养老准备应以制度保障与自我储备为基础,在此之上再将家庭支持视为可能的补充来源。

养老资金准备的核心问题之一在于将退休后预期消费水平与领取期长度转化为可量化的资金需求。示例性演算可用于说明该逻辑。假设个体从30岁开始进行养老准备,工作期约30年;退休后生活期约25年及以上。若退休后每月支出目标为20000元,则年度支出约为24万元;若按25年计算,则累计支出约为600万元。该演算以简单乘法形式展示退休期长度与支出水平对养老资金需求规模的决定性影响,并提示在实际规划中需要进一步考虑收益率、通货膨胀、税收、医疗支出波动与寿命不确定性等因素对资金需求的影响方向。

\begin{table}[htbp]
  \centering
  \caption{养老资金需求示例的数量化演算。数据来源:演示文稿第46页。}
  \label{tab:funding_example}
  \renewcommand{\arraystretch}{1.25}
  \begin{tabular}{p{0.28\textwidth}p{0.60\textwidth}}
    \toprule
    项目 & 计算与结果 \\
    \midrule
    退休后月度支出目标 & 20000元 \\
    年度支出 & $20000\times 12=24$万元 \\
    退休期长度 & 25年以上 \\
    累计资金需求 & $24\times 25=600$万元 \\
    \bottomrule
  \end{tabular}
\end{table}

上述示例的意义在于建立直观的量化框架,即养老资金需求是支出水平与领取期长度的函数,而领取期长度又与寿命水平和退休年龄相关。若退休年龄延后,则领取期可能缩短,资金需求在其他条件不变时相应下降;若寿命水平提升,则领取期延长,资金需求相应上升。对养老金融产品而言,缴费期、积累期与领取期的匹配关系决定了产品现金流结构与精算评估要点;对个体规划而言,越早进行养老准备越有利于利用更长的积累期实现稳健积累,并通过制度工具与市场化产品组合降低养老资金缺口风险。

\section{本章小结}

本章围绕人口老龄化背景下养老保障体系与商业养老保险的发展逻辑展开论述。人口金字塔、出生人口下降、生育意愿变化与寿命改善共同推动老龄化加速,老龄化社会向深度老龄化阶段跃迁并呈现复杂老龄化特征,抚养比上升对养老保障制度的可持续性提出更高要求。养老资金来源的调查结果显示,居民更倾向于依靠自我储备并对养老保障存在一定焦虑,养老需求呈现多元化与长期化特征。

国家层面通过中长期规划提出夯实财富储备与提高社会保障能力的政策框架,并强调加快建立多层次养老保险制度;监管政策进一步明确保险业服务民生保障的方向,提出积极发展第三支柱养老保险、发展商业保险年金、推动专属商业养老保险与丰富银发经济相关产品供给等要求。三支柱养老保障体系在功能分工上形成“保基本、补充保障与自主积累”相结合的结构:第一支柱覆盖广但对人口结构变化敏感;第二支柱在个人账户与市场化投资框架下提供补充保障但覆盖面仍有限;第三支柱通过税收优惠与完全积累机制促进个人长期养老财富准备但仍处于培育阶段。

在商业养老保险层面,商业保险年金与专属商业养老保险通过长期合同安排、稳健积累机制与领取规则设计,为居民提供跨期财务规划工具,并在双账户管理、保证利率与结算利率机制等方面体现养老属性与长期性要求。年金产品定价与盈利性管理强调预定假设、精算平衡关系、利润测试、新业务价值评估以及三差平衡管理框架,体现养老产品长期经营的精算逻辑与风险管理要求。通过养老资金需求示例,本章进一步展示了退休期长度与支出水平对资金准备规模的影响,为后续精算建模、产品设计与资产负债管理的进一步讨论奠定了问题背景与业务语境。
